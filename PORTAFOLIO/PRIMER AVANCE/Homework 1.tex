\documentclass{report}
\usepackage[letterpaper,top=2cm,bottom=2cm,left=3cm,right=3cm,marginparwidth=1.75cm]{geometry}

\title{Universidad Tecnologica de Tijuana}
\author{Acata Jaramillo Gonzalo De Jesus}
\date{January 2024}

\begin{document}
\maketitle

\begin{center}
    \textbf{Homework}
\end{center}

\begin{center}
    PWA Research
\end{center}

\section{Concepts of PWAs}

A Progressive Web Application or PWA is an application software that relies on technologies such as HTML, CSS, and JavaScript.

PWAs are web apps that leverage the latest and most powerful browser technologies. In other words, they combine the best of the web and the best of native apps.

These are web applications that utilize APIs and emerging features of the web browser along with a traditional progressive enhancement strategy to deliver a native app-like user experience for cross-platform web applications.

\section{Features}

\begin{itemize}
\item \textbf{Enhanced Performance:} Thanks to reduced resource usage, PWAs offer significantly better performance compared to traditional apps, with lower loading times, sometimes even instant.
\item \textbf{Offline Functionality:} Undoubtedly, one of the major advantages of this type of application is the ability to use them without being connected to the internet.
\item \textbf{Developer-Friendly:} Developers venturing into the development of PWAs will find that they require much less time and investment to produce, which is undoubtedly a significant added value.
\item \textbf{Minimal Resource Usage:} This results in advantages such as not occupying space in the memory of devices.
\item \textbf{Adaptability:} Most of these applications have a responsive design that allows them to function on computers, tablets, and mobile devices.
\end{itemize}

\section{Differences between SPA, PWA, and Web Applications}

\textbf{SPA (Single Page Application)}
\begin{itemize}
\item Navigation is smoother and faster because only necessary information is loaded.
\item AJAX (Asynchronous JavaScript and XML) is used to update content without reloading the page.
\item Well-known examples: Gmail, Facebook, Twitter.
\end{itemize}

\vspace{12pt}

\textbf{PWA (Progressive Web Application)}
\begin{itemize}
\item Works on any browser.
\item Progressive, meaning it can function across different levels of capability, from basic to modern browsers.
\item Can be installed on the user's device for quick access, even offline.
\item Provides a native-app-like experience with push notifications and access to device hardware.
\item Well-known examples: Twitter Lite, Pinterest, Flipkart.
\end{itemize}

\newpage

\textbf{Web Application}
\begin{itemize}
\item Can refer to both SPAs and more traditional applications with multiple pages.
\item Does not require installation and runs in the browser.
\item Can access online resources and have interactive features.
\item Can be responsive to adapt to different devices.
\item
 Examples include a variety of web applications, from simple interactive pages to more complex applications.
\end{itemize}

\section{Advantages of Progressive Web Applications (PWA)}
\begin{itemize}
\item Optimization of the site and creation of a web application.
\item Accessibility and ability to be downloaded from a web browser (not relying on traditional app stores like Google Play or Apple Store).
\item Provides an optimized user experience on both desktop and mobile devices.
\end{itemize}

\section{Disadvantages of Progressive Web Applications (PWA)}
\begin{itemize}
\item They may be limited in terms of features and functionalities that can be utilized.
\item The visibility of PWAs is lower than that of traditional apps, leading to a higher investment in marketing and positioning.
\item Not all browsers are updated and support PWAs.
\item They do not have access to all native device features, such as Bluetooth.
\item Compatibility with older browsers can become an issue.
\end{itemize}

\end{document}