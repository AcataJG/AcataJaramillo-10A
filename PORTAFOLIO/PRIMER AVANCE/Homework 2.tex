\documentclass{report}
\usepackage[letterpaper,top=2cm,bottom=2cm,left=3cm,right=3cm,marginparwidth=1.75cm]{geometry}

\title{Universidad Tecnologica de Tijuana}
\author{Acata Jaramillo Gonzalo De Jesus}
\date{January 2024}

\begin{document}
\maketitle

\begin{center}
    \textbf{Homework}
\end{center}

\begin{center}
    PWA Tools
\end{center}

\section{Development and Execution Tools}
\textbf{Development and Design}
\begin{itemize}
\item Visual Studio Code
\item Chrome DevTools
\item React, Angular, Vue.js
\end{itemize}

\vspace{12pt}

\textbf{Testing}
\begin{itemize}
\item Lighthouse
\item Jest
\end{itemize}

\vspace{12pt}

\textbf{Deployment and Hosting}
\begin{itemize}
\item Firebase
\item Netlify
\end{itemize}


\section{PWA Installation Requirements}

\begin{itemize}
\item \textbf{PWA Manifest:} The PWA must include a manifest file (manifest.json) with detailed information about the application.
\item \textbf{Service Worker:} The PWA must register a Service Worker to enable offline functionality and manage caches.
\item \textbf{HTTPS:} The PWA must be served over HTTPS to ensure the security of connections.
\item \textbf{Application Icon:} An application icon must be provided for display on the device's home screen or taskbar.
\item \textbf{meta Viewport Tag:} Include a meta viewport tag in the HTML to ensure proper display on mobile devices.
\item \textbf{Compatible Browser:} Ensure that the browser used is compatible with PWAs (e.g., Google Chrome, Mozilla Firefox, Microsoft Edge).
\item \textbf{Installation Permissions:} Users should have the option to install the PWA, typically through an installation banner.
\end{itemize}

\section{What is the best environment for developing PWAs?}

React with Create React App

\end{document}